\documentclass{article}
\usepackage{amsmath, amssymb}

\begin{document}

\section*{Two Item Unbounded Knapsack Problem}

We are given:
\begin{itemize}
    \item An item $A$ with size $s_A$ and cost $c_A$
    \item An item $B$ with size $s_B$ and cost $c_B$
    \item A target size $T$
\end{itemize}

\subsection*{Assumptions}
Without loss of generality, let $A$ be the \emph{most efficient} item, i.e.,
\[
\frac{s_A}{c_A} \geq \frac{s_B}{c_B} \quad \text{or equivalently} \quad \frac{c_A}{s_A} \leq \frac{c_B}{s_B}.
\]

Assume a solution exists. Then it must be of the form:
\[
T = n_A s_A + n_B s_B
\]
for some integers $n_A$ and $n_B$.

Equivalently, we can rewrite:
\[
T - n_A s_A = n_B s_B.
\]

\subsection*{Claim}
We claim that:
\[
n_A = \lfloor T / s_A \rfloor
\]
is the largest value such that $T - n_A s_A$ is divisible by $s_B$.

\subsection*{Proof of Minimal Cost}
Let the cost of this solution be:
\[
C = n_A c_A + n_B c_B.
\]

Assume there exists a different solution with cost:
\[
C' = n_A' c_A + n_B' c_B,
\]
where $C' < C$, and of course:
\[
T = n_A' s_A + n_B' s_B.
\]

Since $A$ is the more efficient item, we must have:
\[
n_A' > n_A.
\]

\paragraph{Proof:}
From $C' < C$, we have:
\[
n_A' c_A + n_B' c_B < n_A c_A + n_B c_B.
\]
Rearranging:
\[
(n_A' - n_A) c_A < (n_B - n_B') c_B.
\]

Substituting $n_B = \frac{T - n_A s_A}{s_B}$ and $n_B' = \frac{T - n_A' s_A}{s_B}$:
\[
(n_A' - n_A) c_A < \frac{(T - n_A s_A) - (T - n_A' s_A)}{s_B} c_B,
\]
\[
(n_A' - n_A) c_A < (n_A' - n_A) \frac{s_A c_B}{s_B}.
\]

Rearranging:
\[
(n_A' - n_A) \left( c_A - \frac{s_A c_B}{s_B} \right) < 0. \tag{1}
\]

From the assumption $\frac{c_A}{s_A} \leq \frac{c_B}{s_B}$, we know:
\[
c_A - \frac{s_A c_B}{s_B} \leq 0.
\]

Thus, for (1) to hold, it must be that:
\[
n_A' - n_A > 0,
\]
i.e., $n_A' > n_A$.

\paragraph{Contradiction:}
If $n_A' > n_A$, then we can write:
\[
T = n_A s_A + (n_A' - n_A) s_A + n_B' s_B,
\]
\[
T - n_A s_A = (n_A' - n_A) s_A + n_B' s_B.
\]

From the construction of $n_A$, we know $T - n_A s_A$ is divisible by $s_B$. Therefore:
\[
(n_A' - n_A) s_A + n_B' s_B
\]
is also divisible by $s_B$. Since $n_A' - n_A > 0$, this contradicts the maximality of $n_A$.

\paragraph{Conclusion:}
The solution with $n_A = \lfloor T / s_A \rfloor$ and $n_B = \frac{T - n_A s_A}{s_B}$ minimizes the cost.

\subsection*{Algorithm}
The algorithm to find the solution is as follows:
\begin{verbatim}
nA = floor(T / sA)
while (T - nA * sA) is not divisible by sB:
    nA -= 1
nB = (T - nA * sA) / sB
\end{verbatim}

\subsection*{Complexity}
If no solution exists, the loop will run more than $s_B$ times. Hence, the time complexity is $O(s_B)$.

Note that this is not yet sufficient to prove that this is poly-time as this is still only
pseudo-polynomial time. 

\subsection*{Making it Log}
However, given a GCD,
Let $F = gcd(s_A,s_B)$

\[
T = n_A s_A + n_B s_B
\]
\[
T/F = n_A s_A/F + n_B s_B/F
\]
We can scale all the solutions by their GCD and still be valid.
Notably, after scaling, $s_A$, $s_B$ must be coprime.
(if $d$ divides $s_A$ and $s_B$ then $d$ must divide $T$. If it does not there is no solution)

Hence the goal is to solve for the largest $n_A$ such that $T-n_A s_A$ is divisible by $s_B$ where $s_A$, $s_B$ coprime.
Or equivalently find the largest $n_A$
\[
T - n_A s_A  = 0 \mod s_B
\]
\[
n_A s_A  = T \mod s_B
\]

Since $s_A$, $s_B$ are coprime, there exists a multiplicative inverse $s_A^{-1}$
which we can solve for with the extended euclidean algorithm:
\[
  s_A s_A^{-1} = 1 \mod s_B
\]

We can solve for a base solution $b$:
\[
  b = T s_A^{-1} \mod s_B
\]
\[
  b s_A  = T \mod s_B
\]

However, to maximize the solution we need to be able to keep adding $0 \mod s_B$.
As $s_A$ and $s_B$ coprime, the only solution is:
\[
  s_B s_A  = 0 \mod s_B
\]

So combining everything, the maximizing $n_A$ is:
\[
n_A = b + \text{floor}(\frac{T - x s_A}{s_A s_B})
\]

The complexity of this is now about $O(\log(T))$

\subsection*{Quicker Nonconstructive Proof}
This can be written as a fixed size integer linear program which has solution 
complexity poly in size (number of bits) of inputs.
\end{document}
